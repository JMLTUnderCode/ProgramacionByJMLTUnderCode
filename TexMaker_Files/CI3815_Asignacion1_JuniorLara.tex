\documentclass[a4paper,12pt]{article}

\usepackage[spanish]{babel}
\usepackage[utf8]{inputenc}
\usepackage[T1]{fontenc}
\usepackage[utf8]{inputenc}
\usepackage{makeidx}
\usepackage{graphicx}
\usepackage{lmodern}
\usepackage{kpfonts}
\usepackage[left=2cm,right=2cm,top=2cm,bottom=2cm]{geometry}
\usepackage{amsmath,amsfonts,amssymb}

\title{Asignación 1}

\begin{document}

\begin{center}
\par \includegraphics[scale=1]{USB} \par
Universidad Simon Bolivar \\ Curso: CI3815 / Organización del Computador \\ Trimestre: Septiembre-Diciembre, 2022 \\ Profesor: Fernando Torre Mora \\ Estudiante: Junior Miguel Lara Torres - Carnet: 17-10303 \\
\end{center}

\begin{center}
Asignación 1
\end{center}

1. ¿Cuántas instrucciones escritas por el programador tiene este programa? Cuenta a partir de la instrucción $ li~~\$t0~~4 $. \\

Este programa contiene un total de 17 instrucciones escritas por el usuario a partir de la instrucción $ li~~\$t0~~4 $. \\

2. ¿Cuántas instrucciones son generadas al cargar este programa en MARS? Cuenta a partir de la instrucción $ li~~\$t0~~4 $. \\

La cantidad de instrucciones generadas por el ensamblador en MARS es de 19. \\

3. ¿Cuántos bytes ocupa el programa en memoria a partir de la instrución $ li~~\$t0~~4 $? Considerar sólo el área de texto. \\

La cantidad de Bytes cargados en memoria al cargar el programa es de 4c en Hexadecimal, lo que equivalentemente es 76 Bytes en Decimal. Así mismo, como tenemos 19 instrucciones generadas por el ensamblador y cada instrucción pesa 4 Bytes, entonces 4*19 = 76 Bytes.\\

4. ¿Qué actividad realiza el programa? Indique cómo llegó a esa conclusión. \\

Se observa un bucle en el programa, tenemos un total de 4 iteraciones, cuyo valor es cargado en registro t0 y que se van reduciendo una vez corre el programa (Linea de código 12). A su vez, si quitamos las lineas 19 y 20 no se carga ninguna información en la RAM al finalizar el programa. Las lineas 22, 23 y 24 permite que al final del programa se almacene la información del carnet en un orden especifico. Las lineas 26 y 27 almacenan dos dígitos del carnet del estudiante. Las lineas 29, 30 y 31 permiten restan 1 al iterador, sumar 2 a los registros que almacenan la dirección del carnet e ing. En particular ese suma 2 me da a entender que estamos recorriendo el carnet de 2 dígitos en 2 dígitos. Por tanto, el programa va recorriendo el carnet de 2 en 2 y guardando dichos digitos en la RAM. \\ \\ \\ \\ \\

5. ¿En qué dirección de memoria se cargó el primer y último dígito de su carnet? \\

Usando el lenguaje ASCII. Se almacenan en la dirección 0x10010000, para el primer dígito se encuentra en la columna VALUE (+14) en el byte número 22 y el último digito en la columna VALUE (+18) en el byte número 28. \\ \\

6. Cargue nuevamente su programa y coloque un breakpoint en la dirección donde se encuentra la etiqueta bp. Ejecute y dé una tabla indicando cómo varían los registros a los largo de la ejecución del programa. Debe incluir todos los registros cuyos valores sufran alguna variación en algún momento de la ejecución del programa. \\

Desactivando la vista en valores Hexadecimales tenemos la siguiente tabla\\

$ \begin{matrix}
   \text{Iteración} & 1 & 2 & 3 & 4 \\
  \$at & 268500992 & 268500992 & 268500992 & 268500992 \\
  \$v0 & 0 & 0 & 0 & 10 \\
  \$t0 & 3 & 2 & 1 & 0 \\ 
  \$t1 & 268500997 & 268500999 & 268501001 & 268501003 \\
  \$t2 & 268501014 & 268501016 & 268501018 & 268501020 \\
  \$s0 & 55 & 48 & 48 & 0  \\
  \$s1 & 49 & 49 & 51 & 51 \\
\end{matrix}  $

\end{document}
