\documentclass[a4paper,12pt]{article}

\usepackage[spanish]{babel}
\usepackage[utf8]{inputenc}
\usepackage[T1]{fontenc}
\usepackage[utf8]{inputenc}
\usepackage{makeidx}
\usepackage{graphicx}
\usepackage{lmodern}
\usepackage{kpfonts}
\usepackage[left=2cm,right=2cm,top=2cm,bottom=2cm]{geometry}
\usepackage{amsmath,amsfonts,amssymb}
\usepackage{hyperref}
\hypersetup{
    colorlinks=true,
    linkcolor=cyan,
    filecolor=magenta,      
    urlcolor=blue,
    pdftitle={titulo},
    pdfpagemode=FullScreen,
    }
\urlstyle{same}

\title{Laboratorio 0 - Asignacion 2}

\begin{document}

\begin{center}
\par \includegraphics[scale=1]{USB} \par
Universidad Simon Bolivar \\ Curso: PS1115 / Sistema de Información I \\ Trimestre: Abril-Julio, 2024 \\ Profesor: Marla Manely Corniel \\ Estudiante: Junior Miguel Lara Torres - Carnet: 17-10303 \\
\end{center}

\begin{center}
Laboratorio 0 - Asignación 2
\end{center}

En media cuartilla, permíteme conocer:

\begin{itemize}
\item ¿De qué están hechos tus sueños?

Mis sueños están hechos de ambición, desarrollo, tener la capacidad de saber que puedo destacar sin que nadie lo sepa, ser un ser diferente (no confundir con superioridad, ego o cualquier otro término que denote estar por encima de) simplemente estar fuera de cualquier saco/categoría/tipo de ser, sin más... ser diferente.

\item ¿Cómo piensas conquistarlos?

Mis sueños o deseos no tienen nada que ver con ser un gran empresario, ser el mejor computista, ser el mejor en lo que hago, etc. El pensamiento crítico y cuestionable de todo lo que vemos, escuchamos y creamos me da por simple inercia todo lo antes mencionado. Para conquistar cualquier cosa solo debo mejorar mi sentido de cuestionamiento, aprender a realizar preguntas correctas, creo en que todo tiene un porqué. ¿Religión o ateísmo? Mi respuesta es: ¿Por qué debo ser solo religioso o solo ateo? ¿Y si no creo en uno u otro, pero si en ambos?, entre millones de preguntas y debates interesantes del pensamiento que pueden surgir, pues la evolución es pensar.

\end{itemize}
\end{document}