\documentclass[a4paper,12pt]{article}

\usepackage[spanish]{babel}
\usepackage[utf8]{inputenc}
\usepackage[T1]{fontenc}
\usepackage[utf8]{inputenc}
\usepackage{makeidx}
\usepackage{graphicx}
\usepackage{lmodern}
\usepackage{kpfonts}
\usepackage[left=2cm,right=2cm,top=2cm,bottom=2cm]{geometry}
\usepackage{amsmath,amsfonts,amssymb}

\title{(3.37) El meme}
\author{Junior Miguel Lara Torres}
\date{today}

\graphicspath {{C:/Users/JuniorLara/Desktop/TexMaker_Files/}}
\begin{document}

\begin{center}
\par \includegraphics[scale=1]{USB} \par
Universidad Simon Bolivar \\ Curso: CI2511 / Logica Simbolica \\ Trimestre: Abril-Julio, 2021 \\ Estudiante: Junior Lara - 17-10303 \\
\end{center}

\begin{center}
El 3.37 Alias El innombrable.
\end{center}

Demuestre: (3.37) Associativity of $\wedge$: $(p \wedge q)\wedge r \equiv p \wedge (q \wedge r)$  \\

Partiendo del lado izquierdo  $(p \wedge q)\wedge r: \\ \\ $ 
$                 ~~~~~~~~(p \wedge q)\wedge r \\ \\
\equiv ~~~~~~~~~~ \left\langle~~ \begin{matrix} 
(3.35)~~ p\wedge q \equiv p \equiv q \equiv p \vee q, \\ 
X=p\wedge q,
~~Y=p \equiv q \equiv p \vee q,
~~E=z \wedge r 
\end{matrix} ~~\right\rangle \\ \\
                 ~~~~~~~~(p \equiv q \equiv p \vee q) \wedge r \\ \\
\equiv ~~~~~~~~~~ \left\langle~~ \begin{matrix} 
(3.35)~~ (p\wedge q \equiv p \equiv q \equiv p \vee q)[p,q:=p \equiv q \equiv p \vee q,~r], \\ 
X=(p \equiv q \equiv p \vee q) \wedge r,
~~Y=p \equiv q \equiv p \vee q\equiv r \equiv (p \equiv q \equiv p \vee q) \vee r,
~~E=z 
\end{matrix} ~~\right\rangle \\ \\
                 ~~~~~~~~p \equiv q \equiv p \vee q\equiv r \equiv (p \equiv q \equiv p \vee q) \vee r \\ \\
\equiv ~~~~~~~~~~ \left\langle~~ \begin{matrix} 
(3.24)~~ (p \vee q \equiv q \vee p)[p,q:=p \equiv q \equiv p \vee q,~r], \\ 
X=(p \equiv q \equiv p \vee q) \vee r,
~~Y=r \vee (p \equiv q \equiv p \vee q),
~~E=p \equiv q \equiv p \vee q\equiv r \equiv z 
\end{matrix} ~~\right\rangle \\ \\
                 ~~~~~~~~p \equiv q \equiv p \vee q\equiv r \equiv r \vee (p \equiv q \equiv p \vee q) \\ \\
\equiv ~~~~~~~~~~ \left\langle~~ \begin{matrix} 
(3.27)~~ (p \vee (q \equiv r) \equiv p \vee q \equiv p \vee r)[p,q,r:=r,~p\equiv q,~p \vee q], \\ 
X=r \vee (p \equiv q \equiv p \vee q),
~~Y=r \vee (p \equiv q) \equiv r \vee (p \vee q),
~~E=p \equiv q \equiv p \vee q\equiv r \equiv z  \end{matrix} ~~\right\rangle \\ \\
                 ~~~~~~~~p \equiv q \equiv p \vee q\equiv r \equiv r \vee (p \equiv q) \equiv r \vee (p \vee q) \\ \\
\equiv ~~~~~~~~~~ \left\langle~~ \begin{matrix} 
(3.27)~~ (p \vee (q \equiv r) \equiv p \vee q \equiv p \vee r)[p,q,r:=r,~p,~q], \\ 
X=r \vee (p \equiv q),
~~Y=r \vee p \equiv r \vee q,
~~E=p \equiv q \equiv p \vee q\equiv r \equiv z \equiv r \vee (p \vee q)  
\end{matrix} ~~\right\rangle \\ \\
                 ~~~~~~~~p \equiv q \equiv p \vee q\equiv r \equiv r \vee p \equiv r \vee q \equiv r \vee (p \vee q) \\ \\
\equiv ~~~~~~~~~~ \left\langle~~ \begin{matrix} 
(3.2)~~ (p \equiv q \equiv q \equiv p)[p,q:=p\vee q,~r], \\ 
X=p \vee q\equiv r ,
~~Y=r \equiv p \vee q ,
~~E=p \equiv q \equiv z \equiv r \vee p \equiv r \vee q \equiv r \vee (p \vee q)
\end{matrix} ~~\right\rangle \\ \\
                 ~~~~~~~~p \equiv q \equiv r \equiv p \vee q \equiv r \vee p \equiv r \vee q \equiv r \vee (p \vee q) \\ \\
\equiv ~~~~~~~~~~ \left\langle~~ \begin{matrix} 
(3.24)~~ (p \vee q \equiv q \vee p)[p,q:=r,~p], \\ 
X=r \vee p ,
~~Y=p \vee r,
~~E= p \equiv q \equiv r \equiv p \vee q \equiv z \equiv r \vee q \equiv r \vee (p \vee q)  
\end{matrix} ~~\right\rangle \\ \\
                 ~~~~~~~~p \equiv q \equiv r \equiv p \vee q \equiv p \vee r \equiv r \vee q \equiv r \vee (p \vee q) \\ \\
\equiv ~~~~~~~~~~ \left\langle~~ \begin{matrix} 
(3.27)~~ p \vee (q \equiv r) \equiv p \vee q \equiv p \vee r~, \\ 
X=p \vee q \equiv p \vee r ,
~~Y=p \vee (q \equiv r),
~~E=p \equiv q \equiv r \equiv z \equiv r \vee q \equiv r \vee (p \vee q)
\end{matrix} ~~\right\rangle \\ \\
                 ~~~~~~~~p \equiv q \equiv r \equiv p \vee (q \equiv r) \equiv r \vee q \equiv r \vee (p \vee q) \\ \\
\equiv ~~~~~~~~~~ \left\langle~~ \begin{matrix} 
(3.2)~~ (p \equiv q \equiv q \equiv p)[p,q:=p \vee (q \equiv r),~r \vee q], \\ 
X=p \vee (q \equiv r) \equiv r \vee q,
~~Y= r \vee q \equiv p \vee (q \equiv r),
~~E=p \equiv q \equiv r \equiv z \equiv r \vee (p \vee q)   
\end{matrix} ~~\right\rangle \\ \\
                 ~~~~~~~~p \equiv q \equiv r \equiv r \vee q \equiv p \vee (q \equiv r) \equiv r \vee (p \vee q) \\ \\
\equiv ~~~~~~~~~~ \left\langle~~ \begin{matrix} 
(3.24)~~ (p \vee q \equiv q \vee p)[p,q:=r,~p\vee q], \\ 
X=r \vee (p \vee q) ,
~~Y=(p \vee q) \vee r,
~~E=p \equiv q \equiv r \equiv r \vee q \equiv p \vee (q \equiv r) \equiv z
\end{matrix} ~~\right\rangle \\ \\
                 ~~~~~~~~p \equiv q \equiv r \equiv r \vee q \equiv p \vee (q \equiv r) \equiv (p \vee q) \vee r \\ \\
\equiv ~~~~~~~~~~ \left\langle~~ \begin{matrix} 
(3.25)~~ (p \vee q) \vee r \equiv p \vee (q \vee r)~, \\ 
X=(p \vee q) \vee r,
~~Y=p \vee (q \vee r),
~~E=p \equiv q \equiv r \equiv r \vee q \equiv p \vee (q \equiv r) \equiv z
\end{matrix} ~~\right\rangle \\ \\
                 ~~~~~~~~p \equiv q \equiv r \equiv r \vee q \equiv p \vee (q \equiv r) \equiv p \vee (q \vee r) \\ \\
\equiv ~~~~~~~~~~ \left\langle~~ \begin{matrix} 
(3.27)~~ (p \vee (q \equiv r) \equiv p \vee q \equiv p \vee r)[q,r:=q \equiv r,~q \vee r], \\ 
X=p \vee (q \equiv r) \equiv p \vee (q \vee r) ,
~~Y=p \vee (q \equiv r \equiv q \vee r),
~~E=p \equiv q \equiv r \equiv r \vee q \equiv z
\end{matrix} ~~\right\rangle \\ \\
                 ~~~~~~~~p \equiv q \equiv r \equiv r \vee q \equiv p \vee (q \equiv r \equiv q \vee r) \\ \\
\equiv ~~~~~~~~~~ \left\langle~~ \begin{matrix} 
(3.24)~~ (p \vee q \equiv q \vee p)[p:=r] , \\ 
X= r \vee q,
~~Y=q \vee r ,
~~E=p \equiv q \equiv r \equiv z \equiv p \vee (q \equiv r \equiv q \vee r)
\end{matrix} ~~\right\rangle \\ \\
                 ~~~~~~~~p \equiv q \equiv r \equiv q \vee r \equiv p \vee (q \equiv r \equiv q \vee r) \\ \\
\equiv ~~~~~~~~~~ \left\langle~~ \begin{matrix} 
(3.35)~~ (p\wedge q \equiv p \equiv q \equiv p \vee q)[q:=q \equiv r \equiv q \vee r], \\ 
X=p \equiv q \equiv r \equiv q \vee r \equiv p \vee (q \equiv r \equiv q \vee r),
~~Y=p \wedge (q \equiv r \equiv q \vee r),
~~E=z
\end{matrix} ~~\right\rangle \\ \\
                 ~~~~~~~~p \wedge (q \equiv r \equiv q \vee r) \\ \\
\equiv ~~~~~~~~~~ \left\langle~~ \begin{matrix} 
(3.35)~~ (p\wedge q \equiv p \equiv q \equiv p \vee q)[p,q:=q,~r], \\ 
X= q \equiv r \equiv q \vee r,
~~Y=q\wedge r,
~~E=p \wedge z
\end{matrix} ~~\right\rangle \\ \\
                 ~~~~~~~~p \wedge (q\wedge r) \\ \\
$ \\

Finalmente por transitividad, $(p \wedge q)\wedge r \equiv p \wedge (q \wedge r)$.

\end{document}