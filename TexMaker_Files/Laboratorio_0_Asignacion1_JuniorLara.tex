\documentclass[a4paper,12pt]{article}

\usepackage[spanish]{babel}
\usepackage[utf8]{inputenc}
\usepackage[T1]{fontenc}
\usepackage[utf8]{inputenc}
\usepackage{makeidx}
\usepackage{graphicx}
\usepackage{lmodern}
\usepackage{kpfonts}
\usepackage[left=2cm,right=2cm,top=2cm,bottom=2cm]{geometry}
\usepackage{amsmath,amsfonts,amssymb}
\usepackage{hyperref}
\hypersetup{
    colorlinks=true,
    linkcolor=cyan,
    filecolor=magenta,      
    urlcolor=blue,
    pdftitle={titulo},
    pdfpagemode=FullScreen,
    }
\urlstyle{same}

\title{Laboratorio 0 - Asignacion 1}

\begin{document}

\begin{center}
\par \includegraphics[scale=1]{USB} \par
Universidad Simon Bolivar \\ Curso: PS1115 / Sistema de Información I \\ Trimestre: Abril-Julio, 2024 \\ Profesor: Marla Manely Corniel \\ Estudiante: Junior Miguel Lara Torres - Carnet: 17-10303 \\
\end{center}

\begin{center}
Laboratorio 0 - Asignaciön 1
\end{center}

Reconocmiento de talentos: Ejercicio exploratorio para asumir reoles en el desarrollo de sistemas.

\begin{itemize}
\item Reconoce, al menos, dos talentos de cada uno de los integrantes de tu equipo... Luego, identificando al compañero(a), señala la razón por la cuál reconoces ese talento. Presenta tus resultados en una cuartilla(un informe por persona).

\begin{itemize}
\item Para Laura Parilli, puedo reconocer que es una chica atenta, con todo aquello que le llama la atención y le gusta sobre todo, esto porque he visto muchas materias con ella y he tenido la oportunidad de ver los niveles de interés al momento de trabajar sobre cualquier tema. Por otro lado, ella comparte, conmigo, un aspecto importante de perfeccionismo en los trabajos, esto sería, el detalle y la organización al momento de realizar una entrega. Esto lo sé porque hemos trabajado juntos para perfeccionar infinidad de trabajos.

\item Para Astrid Alvarado, al igual que Laura, he trabajado y visto materias con ella en muchas oportunidades. He tenido la dicha de conocerla, incluso en persona. Entre los talentos principales puedo destacar que es muy aplicada al desarrollo de un nuevo tema en general, grandes habilidades de investigación y aplicación del contenido investigado, esto último quiere decir que si algo requiere de un entendimiento mayor, por más abstracto que sea, lo puede lograr. Le gusta mucho la programación que radique en lo funcional y bello al máximo, por tanto, el perfeccionismo es un patrón que se nota mucho en sus entregas. Y como dije antes, todo esto lo sé por experiencias vividas junto a ella.

\item Para Yerimar Manzo, no puedo decir mucho dado que es segunda vez que veo una materia con ella y primera vez que trabajo con ella. Sin embargo, puedo notar que el aspecto de liderar un grupo se le da muy bien y es muy atenta. Es relativamente participativa en las actividades de contacto al público, digamos que sabe manejar lo introvertida muy bien, no es un factor determinante al momento de realizar presentación o defensas. Claro está que de este grupo Yerimar es la única con bastante trayectoria en el ámbito laboral, por tanto, el tema comunicacional y de organización se le da muy bien.

\end{itemize}

\end{itemize}

\end{document}