\documentclass[a4paper,12pt]{article}

\usepackage[spanish]{babel}
\usepackage[utf8]{inputenc}
\usepackage[T1]{fontenc}
\usepackage[utf8]{inputenc}
\usepackage{makeidx}
\usepackage{graphicx}
\usepackage{lmodern}
\usepackage{kpfonts}
\usepackage[left=2cm,right=2cm,top=2cm,bottom=2cm]{geometry}
\usepackage{amsmath,amsfonts,amssymb}
\usepackage{hyperref}
\hypersetup{
    colorlinks=true,
    linkcolor=cyan,
    filecolor=magenta,      
    urlcolor=blue,
    pdftitle={titulo},
    pdfpagemode=FullScreen,
    }
\urlstyle{same}

\title{Parcial 1}

\begin{document}

\begin{center}
\par \includegraphics[scale=1]{USB} \par
Universidad Simon Bolivar \\ Curso: CI4325 / Interfaces con el Usuario \\ Trimestre: Enero-Marzo, 2024 \\ Profesor: Franco Gabriel Nori Gonzalez \\ Estudiante: Junior Miguel Lara Torres - Carnet: 17-10303 \\
\end{center}

\begin{center}
Examen 1 (15\%)
\end{center}

\begin{itemize}

\item (7,5 puntos) Seleccione un producto del programa “Dragons Den (UK)” y a partir del mismo identifique lo siguiente:

\textit{\textbf{Video utilizado:}} \url{https://www.youtube.com/watch?v=Y0ylz9hIC7Q}
\begin{itemize}
\item \textbf{Hipótesis}: \\ 
Los oficinistas, profesores y empresas en general pasan mucho tiempo corrigiendo trabajos/informes/libros que se tornan repetitivos y una herramienta que automatice este proceso podría liberar tiempo valioso reduciendo las jornadas hasta un 75\%.

\item \textbf{Mercado}: \\
Oficinas, empresas, universidades y escuelas, en especial los profesores encargados de corregir actividades académicas.

\item \textbf{Problema}: \\
Las personas adscritas a instituciones que pertenezcan a algún departamento que maneje documentación o papeleo un tanto masivos están sujetas a realizar notas/observaciones constantemente. Esta escritura prolonga la carga horaria al necesitar escribir tanto debido a la cantidad de actividades. El profesor en promedio toma 2 a 3 horas para corregir 30 libros de forma manual y las personas a menudo se ven abrumadas por las tareas repetitivas y la sobrecarga de información, lo que puede resultar en una disminución de la productividad y el rendimiento.

\item \textbf{Producto}: \\
Smart Mark es un dispositivo que utiliza tecnología de reconocimiento de voz para ayudar a las personas en el ámbito laboral, acelerando la mecánica de hacer apuntes/correcciones/observaciones. Las personas simplemente ‘dicen’ sus comentarios en voz alta mientras marcan los libros de sus estudiantes, documentos o cualquier otra actividad. Este feedback verbal se dicta directamente en su aplicación móvil y se puede imprimir instantáneamente, ya sea en etiquetas o en hojas de portada A4.

\item \textbf{Posibles Competidores}: \\
Cualquier otra herramienta o software que automatice el proceso de corrección de trabajos podría ser un competidor. Sin embargo, existen varias aplicaciones y herramientas en el mercado que ofrecen soluciones similares como en el espacio de EdTech podrían ser SAM Learning, Alps, Learning Ladders y Showbie. Por otra parte, ya no tan similares pero que podrian relacionarse de alguna forma serian Trello, Asana, Evernote y Microsoft Planner.

\item \textbf{Propuesta de valor (factor diferenciador)}: \\
El factor diferenciador de ‘Smart Mark’ es su uso de la tecnología de reconocimiento de voz para acelerar el proceso de corrección, lo que podría ser una solución única en el mercado y no solo acelera el proceso de marcado, sino que también almacena todo lo que dice el usuario directamente en su libro de calificaciones/apuntes en línea. Esto permite en particular a los profesores construir una imagen muy clara del progreso de sus estudiantes simplemente marcando su trabajo. Adicionalmente, cuenta con una interfaz de usuario innovadora y minimalista, integración con otras herramientas y aplicaciones, y un enfoque en la personalización y la automatización de tareas específicas del usuario. Algunos puntos a tomar en cuenta:
\begin{itemize}
\item \textbf{Corrección de documentos}: Al igual que los profesores, los profesionales en las oficinas a menudo tienen que revisar y corregir documentos. ‘Smart Mark’ podría ayudar a acelerar este proceso.\\

\item \textbf{Accesibilidad}: Para aquellos con discapacidades físicas, una herramienta de reconocimiento de voz podría facilitar la corrección de documentos.
\end{itemize}
\end{itemize}

\item (7,5 puntos) Sobre el producto elegido (sin importar su status actual), elabore un plan de validación en el que incluya brevemente.

\begin{itemize}
\item \textbf{Validación del problema}: Para validar el problema, se puede realizar una encuesta o entrevista a un grupo de profesores para entender cuánto tiempo dedican a la corrección de trabajos y si consideran que este es un problema significativo. También se puede investigar si existen estudios o investigaciones que respalden la afirmación de que los profesores pasan mucho tiempo en la corrección de trabajos. Este estudio/validación se puede ampliar facilmente al area de oficinas y empresas.

\item \textbf{Validación del mercado}: Para validar el mercado, se puede investigar el tamaño total del mercado de profesores e instituciones educativas que podrían beneficiarse de una solución como Smart Mark. Esto podría implicar investigar el número total de personas oficinistras/profesores/empresarios en un área geográfica específica, así como el número de instituciones educativas. También se puede investigar si estas instituciones tienen el presupuesto para invertir en una solución como Smart Mark.

\item \textbf{Validación del producto}: Para validar el producto, se puede realizar una prueba de prototipo en una o varias escuelas para ver cómo los profesores interactúan con Smart Mark y si realmente les ayuda a acelerar su proceso de corrección, tambien para oficinas y empresas interesadas. Se puede recoger feedback de las personas sobre su experiencia con el producto y hacer ajustes en función de este feedback. También se puede medir el tiempo que las personas pasan en sus actividades antes y después de usar Smart Mark para ver si realmente reduce el tiempo de corrección/apuntes/observaciones.
\end{itemize}
\end{itemize}
\end{document}